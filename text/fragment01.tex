
\begin{document}

\section{Computation of the Cohomology}

To compute the cohomology of $Y$, observe that
there is a rational map
\[
   f: Y \dashedarrow \PP^2
\]
given by $f(X_0, \ldots X_4) = (X_0, X_1 , X_2)$.
This map is undefined at the three points
$B = \set{b_1, b_2, b_3}$ of intersection
of the line $X_0 = X_1 = X_2 = 0$ with $Y$.
Let $Y'$ be the variety obtained by blowing
up thepoints of $B$. Then there is a map
\[
   f: Y' \map \PP^2
\] which presents $Y'$ as a fiber space over
$\PP^2$. Over $\PP^2 - E$, the fiber is a
Fermat elliptic curve. Over $E$, the fiber
is a set of three distinct lines passing
through a point. Even more is true: over
$E$, $Y'$ has the structure of a product.


Notice that $H^3(Y) \cong H^3(Y')$. Thus we can compute the first group by computing
the second. We will do this with the help of the Leray spectral sequence for $f$. The
relevant terms are
\[
  E_2^{p,q} = H^p(\PP^2, R^qf_*\ZZ),
\]
Since the fibers of $f$ are
algebraic curves, $E_2^{0,3} = 0$. Since $R^0f_*\ZZ \cong \ZZ$,
$E_2^{3,0} = H^3(\PP^2, \ZZ) = 0$. Thus there are only two
non-vanishisng terms. The first is

\begin{equation}
\label{leray21} E_2^{2,1} = H^2(\PP^2, R^1f_*\ZZ).
\end{equation}

The coefficient sheaf
is supported on $\PP^2 - E$, and on that open set it is a local system with fiber
$\ZZ^2$. The fiber is also a free $\EEE$-module of rank one. Thus the local system is
associated to one of the two non-trivial representations of
\[
  \pi_1(\PP^2 - E) \map Aut(\EEE),
\]
that is, the representation which sends a generator $\gamma$ to $\omega$
or $\bar\omega$, where $\omega = \exp 2\pi \sqrt{-1}/3$. Thus we have

\begin{equation}
\label{eq:locsyscoh} H^2(\PP^2, R^1f_*\ZZ) = H^2(\PP^2 - E, R^1f_*\ZZ).
\end{equation}


where the coefficient sheaf is a canonical local system of rank one $\EEE$-modules,
which we shall write as $\EEE_{\PP^2 - E}$ in recognition of the support of this local
system.

\end{document}
