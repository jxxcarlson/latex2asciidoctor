%%%%%%%%%%%%%%%%%%%



\newcommand{\nb}[1]{\marginpar{\tiny\raggedright\textsf{\hspace{0pt}#1}}}
%\newcommand{\nb}[1]{}

\newtheorem{lemma}{Lemma}

\newcommand{\degree}{\mathop{degree}}

\newcommand{\FIG}[1]{\bigskip \[ FIGURE: \hbox{#1} \] \bigskip }

\newcommand{\set}[1]{\{#1\}}
\newcommand{\sett}[2]{\{#1\ |\ #2\}}

\newcommand{\modulo}{{\ \hbox{mod}\ }}


\newcommand{\map}{\longrightarrow}
\newcommand{\transpose}{\kern1pt{}^t\kern-2.0pt}

\newcommand{\QL}{{\mathbb Q}_\ell}

% \newcommand{\AA}{{\mathbb A}}
\newcommand{\BB}{{\mathbb B}}
\newcommand{\CC}{{\mathbb C}}
\newcommand{\DD}{{\mathbb D}}
\newcommand{\EE}{{\mathbb E}}
\newcommand{\FF}{{\mathbb F}}
\newcommand{\GG}{{\mathbb G}}
\newcommand{\HH}{{\mathbb H}}
\newcommand{\II}{{\mathbb I}}
\newcommand{\JJ}{{\mathbb J}}
\newcommand{\KK}{{\mathbb K}}
\newcommand{\LL}{{\mathbb L}}
\newcommand{\MM}{{\mathbb M}}
\newcommand{\NN}{{\mathbb N}}
\newcommand{\OO}{{\mathbb O}}
\newcommand{\PP}{{\mathbb P}}
\newcommand{\QQ}{{\mathbb Q}}
\newcommand{\RR}{{\mathbb R}}
% \newcommand{\SS}{{\mathbb S}}
\newcommand{\TT}{{\mathbb T}}
\newcommand{\UU}{{\mathbb U}}
\newcommand{\VV}{{\mathbb V}}
\newcommand{\WW}{{\mathbb W}}
\newcommand{\XX}{{\mathbb X}}
\newcommand{\YY}{{\mathbb Y}}
\newcommand{\ZZ}{{\mathbb Z}}

\newcommand{\AAA}{{\mathcal A}}
\newcommand{\BBB}{{\mathcal B}}
\newcommand{\CCC}{{\mathcal C}}
\newcommand{\DDD}{{\mathcal D}}
\newcommand{\EEE}{{\mathcal E}}
\newcommand{\FFF}{{\mathcal F}}
\newcommand{\GGG}{{\mathcal G}}
\newcommand{\HHH}{{\mathcal H}}
\newcommand{\III}{{\mathcal I}}
\newcommand{\JJJ}{{\mathcal J}}
\newcommand{\KKK}{{\mathcal K}}
\newcommand{\LLL}{{\mathcal L}}
\newcommand{\MMM}{{\mathcal M}}
\newcommand{\NNN}{{\mathcal N}}
\newcommand{\OOO}{{\mathcal O}}
\newcommand{\PPP}{{\mathcal P}}
\newcommand{\QQQ}{{\mathcal Q}}
\newcommand{\RRR}{{\mathcal R}}
\newcommand{\SSS}{{\mathcal S}}
\newcommand{\TTT}{{\mathcal T}}
\newcommand{\UUU}{{\mathcal U}}
\newcommand{\VVV}{{\mathcal V}}
\newcommand{\WWW}{{\mathcal W}}
\newcommand{\XXX}{{\mathcal X}}
\newcommand{\YYY}{{\mathcal Y}}
\newcommand{\ZZZ}{{\mathcal Z}}

\newcommand{\fa}{{\frak a}}
\newcommand{\fb}{{\frak b}}
\newcommand{\fc}{{\frak c}}
\newcommand{\fd}{{\frak d}}
\newcommand{\fe}{{\frak e}}
\newcommand{\ff}{{\frak f}}
\newcommand{\fg}{{\frak g}}
\newcommand{\fh}{{\frak h}}
% \newcommand{\fi}{{\frak i}}
\newcommand{\fj}{{\frak j}}
\newcommand{\fk}{{\frak k}}
\newcommand{\fl}{{\frak l}}
\newcommand{\fm}{{\frak m}}
\newcommand{\fn}{{\frak n}}
\newcommand{\fo}{{\frak o}}
\newcommand{\fp}{{\frak p}}
\newcommand{\fq}{{\frak q}}
\newcommand{\fr}{{\frak r}}
\newcommand{\fs}{{\frak s}}
\newcommand{\ft}{{\frak t}}
\newcommand{\fu}{{\frak u}}
\newcommand{\fv}{{\frak v}}
\newcommand{\fw}{{\frak w}}
\newcommand{\fx}{{\frak x}}
\newcommand{\fy}{{\frak y}}
\newcommand{\fz}{{\frak z}}


\newcommand{\mapright}[1]{\ \smash{
   \mathop{\longrightarrow}\limits^{#1}}\ }

\newcommand{\mapleft}[1]{\ \smash{
   \mathop{\longleftarrow}\limits^{#1}}\ }

\newcommand{\del}{\partial}
\newcommand{\delbar}{\bar\partial}




\documentclass[11pt]{amsart}

\newtheorem{proposition}{Proposition} \newtheorem{theorem}{Theorem}
\newtheorem{corollary}{Corollary} \newtheorem{problem}{Problem}
\newtheorem{question}{Question} \newtheorem{example}{Example}
\newtheorem{definition}{Definition}

\usepackage{amssymb} \usepackage{amsthm}

%%% Set depth of table of contents \setcounter{tocdepth}{1}

\newtheorem{remark}{Remark}

%%%%%%%%%%%%%%%%%%%%%%%%%%%%%%%%%%%%%%%%%%%%%%%%%%%%%%%%%%%%%%%%%%%%%% %%%%%%%
\newcommand{\DT}{{\mathbb S}} %\newcommand{\remark}[1]{\noindent{\sl #1}}

%\newcommand{\nb}[1]{}
%%%%%%%%%%%%%%%%%%%%%%%%%%%%%%%%%%%%%%%%%%%%%%%%%%%%%%%%%%%%%%%%%%%%%% %%%%%%%

%%%%%%%%%%%%%%%%%%%%%%%%%%%% %% defs.tex

\input defs.tex

\newcommand{\dashedarrow}{-\to}

%%% end of defs.tex %%%%%%%%%%%%%%%%%%%%%%%%%%%%%%%%%%%%%%

\parskip=5pt

\title{Transcendence of Periods of a Cubic Surface \\ (Working Notes)} \author{James
Carlson and Domingo Toledo }

\date{April 13, 2011. Research partially supported by National Science Foundation Grant
DMS-0600816; the first author also gratefully acknowledges the support of the Clay
Mathematics Institute and of CIMAT. \\ } % \emph{file = special\_periods4.text}

% Activate to display a given date or no

\begin{document} \maketitle

{\bf Abstract.} We (hope to) show that the periods of a doubly cyclic cubic threefold
$Y$ are computable in terms of periods the elliptic curve $E$ which defines $Y$.

 CUT AT LINE 67

\parskip=1pt \tableofcontents

\parskip=5pt

\section{Introduction}


Our aim in this note is investigate the transcendence of periods of a cubic surface. By
periods we mean those of the associated cyclic cubic threefold, as described in
\cite{ACT}. A full, or even a substantial understanding of this question is likely very
far away, since it is not yet resolved for elliptic curves, though there are precise
conjectures in this case \cite{Andre}, \cite{Waldschmitt}. What we undertake,
therefore, is quite modest. Let $F(X_0,X_1,X_2) = 0$ define a cubic curve $E$. Let
\[
   Y = \sett{[X_0, \ldots, X_4] \in \PP^4}{X_3^3 + X_4^3 = F(X_0,X_1,X_2)}
\]
be the
``doubly cyclic'' cubic threefold associated to $E$. We will relate the periods of $Y$
to the periods of $E$.

\end{document}

\section{Computation of the Cohomology}

To compute the cohomology of $Y$, observe that there is a rational map \[ f: Y
\dashedarrow \PP^2 \] given by $f(X_0, \ldots X_4) = (X_0, X_1 , X_2)$. This map is
undefined at the three points $B = \set{b_1, b_2, b_3}$ of intersection of the line
$X_0 = X_1 = X_2 = 0$ with $Y$. Let $Y'$ be the variety obtained by blowing up the
points of $B$. Then there is a map \[ f: Y' \map \PP^2 \] which presents $Y'$ as a
fiber space over $\PP^2$. Over $\PP^2 - E$, the fiber is a Fermat elliptic curve. Over
$E$, the fiber is a set of three distinct lines passing through a point. Even more is
true: over $E$, $Y'$ has the structure of a product.

Notice that $H^3(Y) \cong H^3(Y')$. Thus we can compute the first group by computing
the second. We will do this with the help of the Leray spectral sequence for $f$. The
relevant terms are \[ E_2^{p,q} = H^p(\PP^2, R^qf_*\ZZ), \] Since the fibers of $f$ are
algebraic curves, $E_2^{0,3} = 0$. Since $R^0f_*\ZZ \cong \ZZ$, $E_2^{3,0} = H^3(\PP^2,
\ZZ) = 0$. Thus there are only two non-vanishisng terms. The first is \begin{equation}
\label{leray21} E_2^{2,1} = H^2(\PP^2, R^1f_*\ZZ). \end{equation} The coefficient sheaf
is supported on $\PP^2 - E$, and on that open set it is a local system with fiber
$\ZZ^2$. The fiber is also a free $\EEE$-module of rank one. Thus the local system is
associated to one of the two non-trivial representations of \[ \pi_1(\PP^2 - E) \map
Aut(\EEE), \] that is, the representation which sends a generator $\gamma$ to $\omega$
or $\bar\omega$, where $\omega = \exp 2\pi \sqrt{-1}/3$. Thus we have \begin{equation}
\label{eq:locsyscoh} H^2(\PP^2, R^1f_*\ZZ) = H^2(\PP^2 - E, R^1f_*\ZZ). \end{equation}
where the coefficient sheaf is a canonical local system of rank one $\EEE$-modules,
which we shall write as $\EEE_{\PP^2 - E}$ in recognition of the support of this local
system.

To compute the cohomology $(\ref{eq:locsyscoh})$, consider the surface \[ S =
\sett{(X_0: \ldots : X_3:0) \in \PP^4}{X_3^3 = F(X_0,X_1,X_2)} \] The projection of
$\PP^3$ from the center $(X_0 : X_1 : X_2)$ to the plane $\{X_3=0\}$ defines a cyclic
cover \[ g: S \map \PP^2. \] The fiber of $g$ over $\PP^2 - E$ is a set of three
points, and over $E$ it is a single point. The cyclic transformation cyclically
permutes the three points. Thus we have a short exact sequence \[ 0 \to \ZZ_{\PP^2}\to
g_*\ZZ \to \EEE_{\PP^2 - E} \to 0 \] where the kernel is the constant sheaf, embedded
in $g_*\ZZ $ as the fixed subsheaf of the cyclic group action. We get an exact sequence
\[ 0 \to H^2(\PP^2, \ZZ) \to H^2(S, \ZZ) \to H^2(\PP^2 - E, \EEE) \to 0. \] We
recognize this as the decomposition of $H^2(S,\ZZ)$ into primitive cohomology and the
multiples of the hyperplane class. In particular, we have \[ H^2(\PP^2 - E, g_*\ZZ)
\cong H^2(S,\ZZ)_{prim} \] and so \begin{equation} \label{rankleray21} \dim E_2^{2,1} =
\dim H^2(\PP^2 - E, \EEE) = 6. \end{equation} This is the first non-vanishing
$E_2$-term in the Leray sequence for cohomology of $H^3(Y)$.

Let us consider next the other non-vanishing term , which is given by the cohomology
group \[ E_2^{1,2} = H^1(\PP^2, R^2f_*\ZZ) \] The coefficient sheaf fits into a short
exact sequence with a subsheaf supported on $\PP^2$ and quotient supported on $E$:
\begin{equation} \label{r2sheafdec} 0\to \ZZ_{\PP^2}\to R^2f_*\ZZ \to \EEE_E\to 0.
\end{equation} The subsheaf is a constant sheaf (fundamental class of the Fermat
elliptic curve over $\PP^2 - E$ and the sum of the fundamental classes of the three
$\PP^1$'s over $E$) The quotient $\EEE_E$ is a rank one the trivial bundle of
$\EEE$-modules over $E$, i.e., \[ \EEE_E \cong E\times\EEE. \] Thus $H^1(E, \EEE) =
H^1(E)\otimes \EEE$.

To give generators, write a reference fiber as $L_0 + L_1 + L_2$, where the $L_i$ are
concurrent projective lines and $\omega L_i = L_{i+1}$, where $i + 1$ is considered
modulo 3. Then the difference of fundamental classes, $L_1 - L_0$ defines a generator of

The cohomology of the subsheaf is zero since $H^1(\PP^2,\ZZ) = 0$. The cohomology of
the quotient sheaf is \begin{equation} \label{H1:E:EEE} H^1(E,\EEE) \cong
H^1(E,\ZZ)\otimes_\ZZ\EEE. \end{equation} That is, $H^1(E,\EEE)$ is of rank 2 as an
$\EEE$-module and of rank 4 as a $\ZZ$-module. Consequently \begin{equation}
\label{rankleray12} \dim E_2^{1,2} = \dim H^1(\PP^2, R^2f_*\ZZ) = 4. \end{equation}
Taken together, we have \[ \dim E_2^{2,1} + \dim E_2^{1,2}= 10 = \dim H^3(Y,\ZZ). \]
Consequently, the Leray spectral sequence degenerates at $E_2$. We therefore have an
exact sequence of of Eisenstein modules \begin{equation} \label{H3:decomp} 0 \to
H^2(\PP^2 - E, \EEE) \to H^3(Y) \to H^1(E, \EEE) \to 0. \end{equation} %% \nb{Split
over $\QQ$ but not over $\ZZ$?} We now investigate the nature of the two parts.

%% CUT AT 222

\begin{lemma} The Hodge structure $H^2(\PP^2 - E, \EEE)$ is rigid. \end{lemma}

\begin{proof} We argue that one of the two summands in the preceding decomposition is
rigid. To this end, consider the joint Hodge-eigenvalue decomposition for a cyclic
cubic threefold. It yields the following table, the sum of whose entries is 10, the
dimension of $H^3(Y)$. \[ \begin{tabular}{l|c|c} & (2,1) & (1,2) \\ \hline $\omega$ & 4
& 1 \\ $\bar\omega$ & 1 & 4 \end{tabular} \] This table can be split into tables of sum
6 and 4, respectively, in only two ways. One splitting is \[ \begin{tabular}{l|c|c} &
(2,1) & (1,2) \\ \hline $\omega$ & 2 & 1 \\ $\bar\omega$ & 1 & 2 \end{tabular} %%
\qquad %% \begin{tabular}{l|c|c} & (2,1) & (1,2) \\ \hline $\omega$ & 2 & 0 \\
$\bar\omega$ & 0 & 2 \end{tabular} \] The other is \[ \begin{tabular}{l|c|c} & (2,1) &
(1,2) \\ \hline $\omega$ & 3 & 0 \\ $\bar\omega$ & 0 & 3 \end{tabular} %% \qquad %%
\begin{tabular}{l|c|c} & (2,1) & (1,2) \\ \hline $\omega$ & 1 & 1 \\ $\bar\omega$ & 1 &
1 \end{tabular} \] In both cases there is a summand in which the Hodge decomposition is
the same as the eigenvalue decomposition. This term is rigid.

To decide which of the two cases holds, observe that the isomorphism of
$(\ref{H1:E:EEE})$ is in fact an isomorphism of Hodge structures, with the $\EEE$ on
the right endowed with the Hodge structure of type $(1,1,)$. Therefore the dimension of
the pieces of the Hodge-Eisenstein decomposition are given by the right-hand member of
the second pair of tables above. For the other summand, the Hodge and eigenvalue
decomposition are the same. Therefore the summand $H^2(\PP^2 - E, \EEE)$ is rigid, as
required. \end{proof}

The variation in moduli of the elliptic curve $E$ is reflected in the variation of the
Hodge structure $H^1(E)\otimes\EEE$ as a summand of the Hodge structure $H^3(Y)$.

\subsection{Some guesses}

My guess: the decomposition \[ H^3(Y) \cong H^2(\PP^2 - E, R^1f_*\ZZ) \oplus H^1(E,
\EEE) \] is really \[ H^3(Y) \cong H^2_{prim}(S, \ZZ)\otimes_\EEE H^1(E_\omega) +
H^1(E_\tau)\otimes_\ZZ \EEE(-1), \] where $\EEE(-1)$ is the natural Eisenstein module
of rank one endowed with a trivial Hodge structure of type $(1,1)$. %% \nb{Further
guesses. Write decomposition as $H = H' \oplus H''$. Then $K_{per}(H') =
K_{per}(H^1(E_\omega))$ and $K_{per}(H'') = K_{per}(H^1(E_\omega))
K_{per}(H^1(E_\tau))$ } %% The first summand is three-dimensional as an $\EEE$-module
and six-dimensional as a $\ZZ$-module. The second summand has the structure of a
four-dimaensional $\ZZ$-module or a two-dimensional $\EEE$-module, where the action is
given by $\phi(x,y) = (y , -x -y)$.

This suggests that the periods of $Y$ are (1) periods of the Fermat elliptic curve (2)
periods of the Fermat elliptic curve times periods of $E_\tau$. Periods of the second
type are products of transcendental numbers. Are they transcendental? \nb{They should
be, since they are also abelian integrals.}

\section{Mumford-Tate group}

\begin{proposition} The Mumford-Tate group of a doubly cyclic cubic threefold is given
as follows:

\begin{enumerate}

 \item If $\tau = \omega$, then $MT(T) \cong MT(E_\omega)$ is a 2-torus. This is the
case of the Fermat cubic threefold.

 \item If $\tau \ne \omega$ and $E_\tau$ has complex multiplication, then $MT(T) \cong
MT(E_\omega)\times MT(E_\tau)$ is a 4-torus.

 \item If $\tau \ne \omega$ and $E_\tau$ does not have complex multiplication, then
$MT(T) \cong MT(E_\omega)\times MT(E_\tau) = T\times GL_2$, where $T$ is a 2-torus.

\end{enumerate} \end{proposition}

%% CUT ST 348

Recall the decomposition \[ H^3(T) \cong H^3(\PP^2 - E_\tau, R^1f_* \ZZ) \oplus
H^1(E_\tau, (R^2f_*\ZZ)_{prim}), \] where $E_\tau$ is an elliptic curve with period
$\tau$. The local system $(R^2f_*\ZZ)_{prim}$ is a trivial local system of rank one
$\EEE$-modules regarded as a system of Hodge structures of type $(1,1)$. Fiberwise, it
is spanned by differences of lines. We write it as $\EEE(1,1)$. Then \[ H^3(T) \cong
H^2(S)_{prim} \otimes_\EEE H^1(E_\omega) \oplus H^1(E_\tau,\EEE(1,1)). \] Now the
Mumford-Tate group of a direct sum is contained in the Cartesian product of the
Mumford-Tate groups of the factors, which we call $A$ and $B$ Thus we have \[ MT(
H^3(T) ) \subset MT( A ) \times MT( B ) \] For the first term, we have \[ A =
H^2(S)_{prim} \otimes_\EEE H^1(E_\omega) \cong H^1(E_\omega) ^{\oplus 6}, \] So that \[
MT(A) \subset (MT(H^1(E_\omega))^6). \] If the cubic surface has enough symmetry, then
$MT(A)$ is the diagonal and so \[ MT(A) = MT(H^1(E_\omega)). \] Somehow I think we
always have equality.

Now let's look at $B$. Since the local system is trivial, \[ B \cong
(H^1(E_\tau)(1,1))^2 \] and so \[ MT(T) \subset MT(H^1(E_\tau)(1,1))^2 \] But there are
symmetries of $T$ permuting the lines, so that again the image is the diagonal group
and we have \[ MT(T) = MT(H^1(E_\tau)) \] Thus we have \[ MT(T) \subset
MT(H^1(E_\omega)) \times MT(H^1(E_\tau)). \] We claim that that this in an equality.
Hmmmmm....

\section{The Field of Periods}

Let $X$ be an algebraic variety defined over $\QQ$. Let $H^m_{DR}(X)$ and $H^m_B(X)$ be
the de Rham and Betti cohomologies of $X$ computed with rational coefficients. Let
$\phi_1, \ldots, \phi_n$ and $\gamma^1, \ldots, \gamma^n$ be bases of these two vector
spaces. Let $(P_{ij})$ be the matrix which relates these two bases: \[ \phi_i = \sum
P_{ij}\gamma^j. \] This is the \emph{period matrix}, so called because \[ P_{ij} =
\int_{\gamma_j}\phi_i, \] where $\gamma_1, \ldots, \gamma_n$ is a homology basis dual
to $\gamma^1, \ldots, \gamma^n$. The \emph{field of periods} is the field
$K_{per}(H^m(X))$ generated be the entries of the period matrix.

\noindent One can, however, give a much more specific result which implies the same
remark:

\end{document}

\begin{theorem} Let $T$ be a doubly cyclic cubic threefold with branch curve $E_\tau$,
where $E_\tau$ is an elliptic curve with period $\tau$. Then $K_{per}(T)$ is the
compositum of $K(E_\omega)$ and $K(E_\tau)$. \end{theorem}

\begin{corollary} (a) If $\tau = \omega$, then $T$ is the Fermat cubic threefold and
$K_{per}(T) = K_{per}(E_\omega)$ is a field of transcendence degree two. (b) If $\tau
\ne \omega$, and $E_\tau$ has complex multiplication, then then $K_{per}(T) $ has
transcendence degree four. (c) If $\tau \ne \omega$, and $E_\tau$ does not have complex
multiplication, then then $K_{per}(T) $ has transcendence degree at most six;
conjecturally it has transcendence degree exactly six. \end{corollary} \nb{Compare with
$MT$. Note that for generic cubic threefold, tr. deg $\le 100$, for generic cyclic
cubic threefold, tr. deg. $\le 25$.}

\begin{remark} \rm {According to Ribet \cite{Ribet:DF}, the transcendence degree of the
field of periods of a simple abelian variety of dimensions $g$ with complex
multiplication s bounded below by $2 + \log_2 g$. This bound applies in case (c) of the
preceding theorem. Thus the transcendence degree in that case is bounded below by
$4.3$. Consequently the transcendence degree is either 5 or 6.} \nb{Can we decide this
and get a sharp bound?} \end{remark}

\noindent The inequalities of the corollary also follow from the results on the
Mumford-Tate group, coupled with Deligne's result XXX.

\subsection{Towards a proof of the theorem}

Let $H_{DR}$, $H_B$ be the deRham and Betti cohomologies of an algebraic variety.
Suppose that there are given direct sum decompositions $H_{DR} = H'_{DR} \oplus
H''_{DR}$ and $H_{B} = H'_{B} \oplus H''_{B}$. Let \[ \iota: {H_{DR}}\otimes_\QQ \CC
\mapright{\cong} {H_B}\otimes_\QQ \CC \] be the comparison isomorphism. (see
\cite{Deligne:HCA}). The decompositions are said to be \emph{natural} (with respect to
the comparison isomorphism) if \[ \iota( H'_{DR}\otimes_\QQ \CC ) = H'_{B}\otimes_\QQ
\CC . \] and \[ \iota( H''_{DR}\otimes_\QQ \CC ) = H''_{B}\otimes_\QQ \CC . \]

\begin{lemma} Suppose give an natural decomposition of cohomology $H = H' \oplus H''$.
Then \[ K_{per}(H) = K_{per}(H')K_{per}(H'') \] where the right-hand side is the
compositum, i.e., the smallest field containing the two given fields. \end{lemma}

\begin{proof} The assertion is evident: take bases $\phi_i$ and $\gamma_j$ are adapted
to these decompositions, so that $\phi_1, \ldots, \phi_a$ span $H'_{DR}$ and
$\phi_{a+1}, \ldots, \phi_n$ span $H''_{DR}$, and similarly for the $\gamma_j$. Then
the period matrix has the block decomposition \[ P = \left( \begin{array}{cc} P' & 0 \\
0 & P'' \end{array} \right) \] \end{proof}

%% \nb{Idea of proof: the comparison isomorphism is natural whenever the decomposition
comes from the Leray spectral sequence.}

\begin{proposition} Let $X$ and $Y$ be varieties defined over $k$ and let $f: X \map Y$
be a $k$-morphism. Fix $n > 0$. Let $p$, $q$ be such that $n = p + q$. Suppose that
there are just two such pairs $(p,q)$ such that $H^p(Y, R^qf_* k)$ is nonzero. Call
these pairs $(p', q')$ and $(p'', q'')$. Then \[ H^{p+q}(X) = H^{p'}(Y, R^{q'}
{\kern-2pt} f_*k) \oplus H^{p''}(Y, R^{q''} {\kern-3pt} f_*k) \] given by the Leray
spectral sequence is a natural decomposition. \end{proposition}

\begin{proof} The comparison isomorphism is the map in hypercohomomology defined by the
map of complexes $\iota: \Omega_{X/k}^\bullet \map \Omega_{X^{an}}^\bullet $ that maps
a rational $p$-form on a Zariski open set to the same form on the same open set in the
analytic topology. The Leray spectral sequence for $f$ is the Grothendieck spectral
sequence for composition of functors $\Gamma_X = \Gamma_Y \circ f_*$, This spectral
sequence applies to (a) constant coefficients $k$, $\CC$, etc., (b) the de Rham complex
over $k$, $\CC$, etc., viewed as an object in the derived category isomorphic to $k$,
$\CC$, etc. The comparison map on the level of complexes of sheaves gives a comparison
map on the level of hypercohomology and also on the $E_2$ terms of the Leray spectral
sequence. The last assertion concludes the proof. \end{proof}

\section{Computation of Integrals}

Now comes the hard part!

\section{Symmetry Considerations}

We now use symmetry considerations to derive what the sequence $(\ref{H3:decomp})$ must
be. Let $\Lambda$ be the 5-dimensional Eisenstein lattice with its hermitan form of
signature $(4,1)$. The periods of cyclic cubic surfaces give totally geodesic disks in
$B^4$ which are fixed by a certain isometries of order $3$. Any two such disks are
equivalent under the action of the group $\Gamma$.

We will work with one specific disk $D\subset B^4$ chosen as follows. First we use our
knowledge of a representative $P_0 = (2-\bar\omega:1:1:1:1)$ for the period of the
Fermat cubic surface, and we will choose our disk $D$ so that $P_0\in D$. We will then
choose a particular transformation $\tau$ of order $3$ that fixes $D$ ($\tau$ and $D$
determine each other, correspond choosing one of the several disks of cyclic surfaces
that pass through $P_0$).

Next, it is known that $\Gamma_\tau$, the centralizer of $\tau$ in $\Gamma$, is a group
of order $54$ that also fixes $D$ pointwise. This is the group $G_{54}^9$ of
\cite{Hosoh}, or $IV$ of \cite{Dolgachev} or XXX of \cite{Segre}. See these references
for more information.

 Moreover there is an element $\eta\in\Gamma$ of order $4$ that normalizes the group
$\Gamma_\tau$ and that, together with $\Gamma_\tau$ generates a subgroup of order $108$
of $\Gamma$, namely the group of automorphisms of the triple cover of the Gaussian
elliptic curve. This is the group $G_{108}^9$, $III$, or XXX of the above references.
The restriction of $\eta$ to $D$ has an isolated fixed point $P_1$ which represents the
period of this Gaussian surface.

 To find these structures explicitly, we begin with $P_0 = (2 - \bar\omega :1:1:1:1)$
which is more convenient to write as $P_0 = (1- \theta\omega :1:1:1:1)$, and we'll let
$V_0 = (1-\theta\omega,1,1,1,1)\in\Lambda$ representing $P_0$. Note that $h(V_0) = -3$.

 We next find a basis for the sublattice $V_0^\bot \subset\Lambda$. It is easy to see
that the vectors $(1,1,1,-\omega,0), (1,1,1,0,-\omega), (1,1,0,1,-\omega),
(1,0,1,1,-\omega)$ are in $V_0^\bot\cap\Lambda$, are linearly independent, and their
Gram matrix has determinant $3$. Therefore they form a basis for $V_0^\bot\cap\Lambda$.
Calling these vectors $r_1,\dots,r_4$, it is easy to find all $\EEE$-integral linear
combinations of them that have norm $2$, there are $108$ of them, which fall into
$108/6 = 18$ classes up to multiplication by units in $\EEE$. Select one representative
from each class and call them $r_1,\dots,r_{18}$, where the firs $4$ are as already
chosen. Since $h(r_i) = 2$, the reflections $R_i$ in these vectors preserve $\Lambda$,
thus are elements of $\Gamma$ and in fact generate the group $\Gamma_{P_0}$ of order
$648$ of automorphisms of the Fermat cubic surface.

 The Fermat cubic surface is cyclic in several ways, meaning that there are several
geodesic disks of cyclic surfaces meeting at $P_0$. Each one of them is fixed by a
cyclic transformation $\tau$ with eigenvalues $\omega$ wiht multiplicity $3$ and $1$
with multiplicity $2$. This is in the conjugacy class denoted $3 A_2$ in
\cite{Dolgachev, Hosoh}. Some experimentation gives a product of the above reflections
$R_i$ that lies in this conjugacy class. For example, if we let $r_5 = (0,0,1,-1,0)$
then $\tau = R_1R_2R_5R_1R_2R_5$ is in the desired conjugacy class. Explicitly

\begin{equation} \label{eq-tau} \tau = \left(\begin{array}{ccccc} 2 - \omega & \omega -
1 & -1 & -1 & -1 \\1 -\omega & \omega & -1 & -1 & -1 \\ -\omega & \omega & \omega & 0 &
0 \\ -\omega & \omega & 0 & \omega & 0 \\ -\omega & \omega & 0 & 0 & \omega
\end{array}\right) \end{equation} It is easy to check that $r_1, r_2, r_3$ form an
$\EEE$-basis for $\Lambda_\omega$, the intersection with $\Lambda$ of the
$\omega$-eigesnpace of $\tau$. The Gram matrix has determinant $3$ and the lattice
$\Lambda_\omega$ is isometric to the (dual) $D_3(\theta)$ lattice in the notation of
XXX. \nb{check notation and reference}

Similarly we find that $\Lambda_1$, the one-eigenlattice of $\tau$, has basis
$(-\theta\omega,0,1,1,1)$ and $(1,1,0,0,0)$. Note that the sum of these two vectors is
the vector $V_0$ above which represents the period $P_0$ of the Fermat cubic. The Gram
matrix of this basis has deteminant $-3$. Thus the sublattice
$\Lambda_1\oplus\Lambda_\omega\subset\Lambda$ has index $9$. \nb{check the index!} This
decomposition roughly corresponds to the exact sequence \ref{H3:decomp}. MAKE THIS MORE
PRECISE!!!

It is easy to check that the lattice $\Lambda_\omega$ has $54$ vectors of norm $2$,
which fall into $54/6 = 9$ equivalence classes modulo multiplication by units in
$\EEE$. It will be convenient to choose one representative from each class and to label
them $v_1,\dots ,v_9$ given by the correspoding rows of the following matrix:

\begin{equation} \label{eq-roots} \left(\begin{array}{ccccc}1 & 1 & 1 & -\omega & 0 \\1
& 1 & -\omega & 0 & 1 \\1 & 1 & 1 & 0 & -\omega \\0 & 0 & 1 & -1 & 0 \\0 & 0 & 1 & 0 &
-1 \\0 & 0 & 0 & 1 & -1 \\1 & 1 & -\omega & 1 & 0 \\1 & 1 & 1 & 0 & -\omega \\1 & 1 & 0
& -\omega & 1\end{array}\right) \end{equation} The reason for this choice of numbering
is to make it compatible with various properties of the points of inflection of the
corresponding cubic curves. This sublattice represents a marking of the points of
inflection, and the the nine vectors are in one to one correspondence with these
points. The corresponding reflections represent the minus map on the cubic curve
centered at the corresponding point of inflection.

More precisely, if we choose $v_1$ to correspond to the origin of the cubic curve, then
the vectors $v-v2$ and $v1v4$ give a basis for the $\FF_3$-vector space of points of
inflection. If the curve in question has an automorphism of order $4$, then it would
correspond to the permutation $\{1,7,4,2,8,5,3,9,6\}$ of the ($\EEE$-lines) of the
vectors $v_i$. Thus, to find the automorphism $\eta$ of $\Lambda$, we first find an
automorphism of $\Lambda_\omega$ that permutes the $v_i$ in the same way, up to
multiples by units. Some experimentation gives that the linear transformation of
$\Lambda_\omega$ which, in the basis $v_1,v_2,v_4$ has matrix \begin{equation}
\label{eq-orderfour} \left(\begin{array}{ccc}1 & -\omega & 0 \\0 & 0 & 1 \\0 & -1 &
0\end{array}\right) \end{equation} is an isometry of $\Lambda_\omega$ that sends $v_1$
to itself, sends $v_2, v_4$ to a multiples by units of $v_7, v_2$ respectively, thus it
induces the desired permutation of order $4$ of the lines $\EEE v_1,\dots,\EEE v_9$.

We need to extend this transformation to an automorphism of $\Lambda$. To this end, we
first extend it to an automorphism of the sublattice $\Lambda_1\oplus\Lambda_\omega$,
and then show that it actually extends to an automorphism of $\Lambda$.

It will be convenient to use the basis of $\Lambda_1\oplus\Lambda_\omega$ given by the
columns of the following matrix $B$. Note that the first two columns are a basis for
$\Lambda_1$, and the last three the above basis for $\Lambda_\omega$: \begin{equation}
\label{eq-basis} B = \left(\begin{array}{ccccc} -\theta & 1 & 1 & 1 & 0 \\0 & 1 & 1 & 1
& 0 \\\bar\omega & 0 & 1 & -\omega & 1 \\ \bar\omega & 0 & -\omega & 0 & -1 \\
\bar\omega & 0 & 0 & 1 & 0\end{array}\right) \end{equation}

 Let us take the transformation $T$ of $\Lambda_1\oplus\Lambda_\omega$ which, in this
basis, is given by $90^0$ rotation on $\Lambda_1$ and the above transformation
(\ref{eq-orderfour}) on $\Lambda_\omega$, namely

\begin{equation}
\label{eq-splittransformation} T = \left(\begin{array}{ccccc}0 & -1 & 0 & 0 & 0 \\ 1 &
0 & 0 & 0 & 0 \\0 & 0 & 1 & -\omega & 0 \\0 & 0 & 0 & 0 & 1 \\0 & 0 & 0 & -1 &
0\end{array}\right) \end{equation}

This defines an automorphism of
$\Lambda_1\oplus\Lambda_\omega$ of order $4$. Then $\eta = BTB^{-1}$ is the matrix of
this transformation in the standard basis for $\Lambda$. It is an isomorphism of
$\Lambda\otimes_\EEE \QQ(\sqrt{-3})$, and it is an automorphism of $\Lambda$ if and
only if its matrix has $\EEE$- integral entries. We find that \begin{equation}
\label{eq-integrality} \eta = \left(\begin{array}{ccccc} \omega + \theta & -\omega &
1-\omega & -\omega & -\omega \\ \omega & -\omega & 1 & 0 & 0 \\ -\bar\omega & 0 &
-\omega & 0 & \bar\omega \\ \theta & -\omega & -\omega & -\omega & -\omega \\
-\bar\omega & 0 & -\omega & \bar\omega & 0\end{array}\right) \end{equation} which is
indeed an $\EEE$-integral matrix, so $\eta$ is an automorphism of $\Lambda$.

Define a linear map $\Phi:\CC^2\to \CC^{(4,1)}$ by \begin{equation}
\label{eq-embedding} \Phi(u_0,u_1) = (u_0 - \theta u_1,u_0,\bar\omega u_1,\bar\omega
u_1,\bar\omega u_1) \end{equation} Note that $\Phi^* h (u_0,u_1) = 2\sqrt{3}\ \Im{ (
\bar u_0, u_1)}$, thus $\Phi$ is a totally geodesic embedding of the upper half-plane
$\Im{(\bar u_0, u_1)} >0 $ in $B^4$ with image the fixed point set of the cyclic
transformation $\tau$. Observe that $\Phi(1:\omega) = (1 - \theta\omega : 1:1:1:1) =
P_0$, the period of the Fermat cubic surface, and $\Phi(1:i) =
(1+\sqrt{3}:1:i\bar\omega:i\bar\omega:i\bar\omega) = P_1$. Thus $P_1$ is a candidate
for the period of the triple cover of the Gaussian elliptic curve. This will be the
case if and only if it is fixed by a transformation in $\Gamma$ in the conjugacy class
of $\eta$, because this is equivalent to saying that the corresponding surface has
automorphism group the group $G_{108}^9$ of order $108$ mentioned above.

One finds that $\eta((1+\sqrt{3}:1:i\bar\omega:i\bar\omega:i\bar\omega)) = i
(1+\sqrt{3}:1:i\bar\omega:i\bar\omega:i\bar\omega)$, thus $P_1$ is a fixed point of
$\eta$. Thus (\ref{eq-embedding}) is a parametrization of the image of the Hesse pencil
of elliptic curves. BY SOME SORT OF UNIQUENESS STATEMENT THIS SHOULD COMPLETE THE PROOF
OF THE FOLLOWING THEOREM:

\begin{theorem} There exists a universal constant $c$ so that the period vector of the
cyclic cubic surface branched over the cubic curve with periods $(u_0,u_1)$ is given by
$c\Phi(u_0,u_1)$.

\end{theorem}

%% OK TO HERE

\begin{corollary} The transcendence properties of the periods of cyclic cubic surfaces
are the same as those of their branch loci. \end{corollary}

\begin{remark} One finds that $\eta$ has eigenvalues $i$, $-i$ (both of multiplicity
$2$) and $1$ (of multiplcity one). The signature of the $i$-eigenspace is $(1,1)$ while
all the others are positive. Thus the fixed point set of $\eta$ in $B^4$ is a totally
geodesic disk $D_1$ that represents the family XXX written down by Naruki, all of which
have a symmetry of order $4$, see also XXX. The disk $D_1$ meets our disk $D$ of cyclic
cubic surfaces in the period $P_1$ of the cyclic cover of the Gaussian curve.
\end{remark}

\begin{remark} The computation of the constant $c$ requires the computation of
integrals. \end{remark}



\vfill\eject

\begin{thebibliography}

\bibitem{ACT} D. Allcock, J. Carlson, and D. Toledo, The complex hyperbolic geometry of
the moduli space of cubic surfaces {\sl J. Algebraic Geom}. 11 (2002), 659-724.
(math.AG/0007048)

\bibitem{Andre} Y. Andre, Galois theory, motives and transcendental numbers,
arXiv:0805.2569v1 [math.NT]

\bibitem{Deligne:HCA} Hodge cycles on abelian varieties (the notes of most of the
seminar ``P\'eriodes des Int\'egrales Ab\'e?liennes'' given by P. Deligne at I.H.E.S.,
197879; pp 9100 of Deligne et al. 1982,

\bibitem{Deligne:CHAPI} Deligne, Pierre, Cycles de Hodge absous et p\'eriodes des
int\'egrales des vari\'et\'es ab\'eliennes. (R\'edig\'es par J. L. Brylinski),
\emph{M\'emoires de la Soci\'et\'e Math\'ematique de France}, $2^e$ s\'erie, tome 2
(1980), p. 23-33

\bibitem{Dolgachev} I. Dolgachev, V. Iskovskikh, Finite subgroups of the plane Cremona
group, in {\emph Algebra, arithmetic, and geometry: in honor of Yu. I. Manin}. Vol. I,
443548, Progr. Math., 269, Birkhuser Boston, Inc., Boston, MA, 2009.

\bibitem{Hosoh} T. Hosoh, Automorphism groups of cubic surfaces, Jour of Alg {\bf 192}
(1997), 651--677.

\bibitem{Ribet:DF} Ribet, K, Divisoin fields of abelina varieties with complex
mulitplication, \emph{M\'emoires de la Soci\'et\'e Math\'ematique de France}, $2^e$
s\'erie, tome 2 (1980), p. XX-XX.

\bibitem{Segre} B. Segre, {\emph The Non-singular Cubic Surfaces}, Oxford University
Press, 1942.

\bibitem{Waldschmitt} M. Waldschmitt, Transcendence of Periods: The State of the Art,
{\sl Pure and Applied Mathematics Quarterly} Volume 2, Number 2 (Special Issue: In
honor of John H. Coates, Part 2 of 2) 435--463, 2006

\end{thebibliography}

\vskip0.30truein \obeylines \parskip=0pt James A. Carlson: jcarlson AT claymath.org
Clay Mathematics Institute \bigskip Domingo Toledo: toledo AT math.utah.edu Department
of Mathematics University of Utah

\end{document}

%%% BAD INPUTS

%% (1)

\begin{remark}{Can we use some kind of Hodge-theoretic semisimplicity to show a priori
that integrals like this are products? I.e., whenever they come from a tensor product?}
\end{remark}

%% (2)

\begin{thebibliography}{99}
