
\begin{document}

\title{Counting Maps of Riemann Surfaces: \\ Hurwitz Theory for Undergraduates}

\author{Renzo Cavalieri and Eric Miles}

\date{}

\maketitle



%\todo[inline]{change style back to amsart from article}
%\todo[inline]{what style should this be in? does it matter?}

%\begin{abstract}
%Here we have a book.
%\end{abstract}

\tableofcontents

%\listoftodos


%%%%%%%%%%%%%%%%%%%%%%%%%%%%%%%%%%%%%%%%%%%%%%%%%%%%%%%%%%%%
%%%%%%%%%%%%%%%%%%%%%%%%%%%%%%%%%%%%%%%
%%%%%%%%%%%%%%
\chapter{Introduction}
\label{introduction}





%%%%%%%%%%%%%%%%%%%%%%%%%%%%%%%%%%%%%%%%%%%%%%%%%%%%%%%%%%%%
%%%%%%%%%%%%%%%%%%%%%%%%%%%%%%%%%%%%%%%
%%%%%%%%%%%%%%

\chapter{From Complex Analysis to Riemann Surfaces}
\label{complexAnalysis}



\begin{theorem}[Open Mapping Theorem]
\label{openMappingThm}
\end{theorem}



\begin{theorem}[Inverse Function Theorem]
\label{inverseFunctionThm}
\end{theorem}






%%%%%%%%%%%%%%%%%%%%%%%%%%%%%%%%%%%%%%%%%%%%%%%%%%%%%%%%%%%%
%%%%%%%%%%%%%%%%%%%%%%%%%%%%%%%%%%%%%%%
%%%%%%%%%%%%%%

\chapter{Introduction to Manifolds}
\label{manifolds}

In Chapter \ref{complexAnalysis} we created honest domains for the functions $\log z$ and $z^{1/2}$ by ``gluing together'' copies of $\bC$. Manifolds generalize this construction and capture the idea of a geometric space $A$ being locally indistinguishable from a space $B$, but globally is (or at least, can be) significantly different from $B$.




\section{General Definition of a Manifold}

An illustration to have in mind is how the earth can locally be represented on your flat computer screen in
Google Maps, but globally the earth is (SPOILER ALERT!) roughly spherical.




Suppose you are Google-mapping your neighboorhood with your house at the center of your screen.
We think of the map you are looking at as a function
$$
  \varphi:(\text{a subset of the earth}) \to (\text{your flat computer screen}).
$$

\end{document}
